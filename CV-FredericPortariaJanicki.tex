\documentclass[margin,line]{res}

\usepackage[utf8]{inputenc}
\usepackage{paralist}
\usepackage{hyperref}

\oddsidemargin -.5in
\evensidemargin -.5in
\textwidth=6.0in
\itemsep=0in
\parsep=0in
% if using pdflatex:
%\setlength{\pdfpagewidth}{\paperwidth}
%\setlength{\pdfpageheight}{\paperheight} 

\newenvironment{list1}{
  \begin{list}{\ding{113}}{%
      \setlength{\itemsep}{0in}
      \setlength{\parsep}{0in} \setlength{\parskip}{0in}
      \setlength{\topsep}{0in} \setlength{\partopsep}{0in} 
      \setlength{\leftmargin}{0.17in}}}{\end{list}}
\newenvironment{list2}{
  \begin{list}{$\bullet$}{%
      \setlength{\itemsep}{0in}
      \setlength{\parsep}{0in} \setlength{\parskip}{0in}
      \setlength{\topsep}{0in} \setlength{\partopsep}{0in} 
      \setlength{\leftmargin}{0.2in}}}{\end{list}}


\begin{document}

\name{Frédéric Portaria-Janicki, B. Ing.\vspace*{.1in}}

\begin{resume}
\section{\sc Information}
\vspace{.05in}
\begin{tabular}{@{}p{2in}p{4in}}
4822 rue Saint-André& {\it Mobile:}  (514) 969-9816 \\            
Montréal, Quebec, H2J 3A3 &{\it Courriel:} frederic@janicki.ninja\\      
\end{tabular}

%%%%%%%%%%%%% FORMATION ACADEMIQUE %%%%%%%

%%%%%%%%%%%%%% ETS Genie Electrique %%%%%%%%%

\section{\sc Formation Académique}
{\bf Baccalauréat en Génie Électrique et Informatique} \hfill {\bf 2015}\\
École de technologie supérieure (ÉTS)
Université du Québec, Montréal

%%%%%%%%%%%%%% Cegep %%%%%%%%%%%%%%%

{\bf Diplôme d'études collégiales en sciences de la nature} \hfill {\bf 2010}\\
Collège Jean-de-Brébeuf

%%%%%%%%%%%%%%% TRAVAIL %%%%%%%%%%%

\section{\sc Expérience professionnelle}
{\bf nventive} \hfill {\bf 2015}\\
Développeur Mobile - Ingénierie de développement logiciel\\\\
Développement multiplateformes (Android/iOS/Windows Phone) d'applications mobiles pour clients sous Xamarin.
{\emph Exemples de clients:} Rdio, Twitter.

{\emph Technologies utilisées:} C\#, Xamarin, .NET 5, Android SDK, iOS, Git, Visual Studio Online.

%\begin{compactitem}
%\item Programmation orientée objet en langage C\#
%\end{compactitem}

{\bf Mobeewave} \hfill {\bf 2015}\\
Stagiaire - Ingénierie de développement logiciel\\\\
Participation au développement d'une application de type serveur back-end en ajoutant de nouvelles fonctionnalitées et en participant au processus de design avec une méthodologie {\emph Agile}.
{\emph Accomplissements:} Design et implémentation d'un système de cache partagée. 

{\emph Technologies utilisées:} C\#, .NET 4.5, Amazon AWS, Redis, WebApi, Git, TFS, Visual Studio Online.

%\begin{compactitem}
%\item Programmation orientée objet en langage C\#
%\end{compactitem}

{\bf iBwave} \hfill {\bf 2014}\\
Stagiaire - Ingénierie de développement logiciel\\\\
Participation au développement de l'application {\emph iBwave design} en ajoutant de nouvelles fonctionnalitées et en participant au processus de design.

{\emph Technologies utilisées:} C\#, XML, .NET 4.5

%\begin{compactitem}
%\item Programmation orientée objet en langage C\#
%\end{compactitem}

{\bf Hewlett-Packard} \hfill {\bf 2012-2013}\\
Stagiaire - Ingénierie de test sur produits HP \\
\\
{\emph Habiletés développées: }
\begin{compactitem}
\item Connaissance des technologies de réseau d'entreprise.
\item Programmation .NET, Powershell, Shell, Git.
\end{compactitem}

{\bf CGI} \hfill {\bf 2008-2012}\\
Consultant
\begin{compactitem}
\item Diagnostic et résolution de problèmes techniques.
\item Support technique informatique pour clients corporatifs.
\end{compactitem}

%%%%%%%%%% Connaissances Particulieres %%%%%%%%

\section{\sc Connaissances Particulières}

{\bf Informatique} \\
Langages de programmation: C, C\#, Xamarin, Android, iOS, GLSL, Java\\
Logiciels de design et principes: SOLID, REST, Unity 3D, Visual Studio 2013, TouchDesigner, \LaTeX \\\\
{\bf Électronique} \\
Électronique numérique, conception matérielle, programmation de microcontrôleurs, VHDL

%%%%%%%%%%%% AUTRES EXP %%%%%%%%%%%%%%

\section{\sc Expérience\\ Divers}
{\bf Projets}\hfill
\begin{itemize}
\item Design et construction d'un dôme géodesique dans le contexte d'un projet multidisplinaire artistique pour le festival OpenMind 2015. Projet récipiendaire d'une bourse pour son design écologique.
\item Design et fabrication d'un mur de 500 LED avec contrôle interactif numérique. Présenté lors d'une soirée de Noël chez nventive ainsi qu'une soirée résautage organisée par Facebook.
\\\\{\emph Technologies utilisées:} TouchDesigner, ARM Cortex-M4, C, Communication série, Découpage laser, Design electrique.
\item Développement d'une plateforme me permettant de faire de la génération d'images 2D et 3D en temps-réel pour concerts de musique. (VJing).
\end{itemize}
{\bf Implication} \hfill
\begin{itemize}
\item Directeur technique du collectif d'artistes {\emph Eden Creative}.
\item Membre du club de développement de jeux vidéo {\emph Conjure}.
\item Développement d'un prototype de jeu multijoueur dans le cadre du concours Ubisoft Académia 2014. Prototype gagnant de 2 des 10 prix disponibles.
\item Participation à la compétition de sécurité informatique NorthSec 2014.
\item Bénévole lors des jeux de Génie 2012.
\end{itemize}
{\bf Entrepreunariat} \hfill{\bf 2010-2012}\\
Conception et mise en marché d'un produit électronique servant à la modification et réparation de téléphones cellulaires \emph{Samsung Galaxy S}
%%%%%%%%%%% Mentions%%%%%%%%%%%%%%%%%

%%%%%%%%%%% LOISIRS %%%%%%%%%%%%%%%%%

\section{\sc Loisirs}
Escalade - Yoga - VJing -  Informatique - Art visuel procédural - Photographie - Vélo

%%%%%%%%%%%% FIN %%%%%%%%%%%%%%%%

\end{resume}
\end{document}




